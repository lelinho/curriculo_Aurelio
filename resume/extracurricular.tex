%-------------------------------------------------------------------------------
%	SECTION TITLE
%-------------------------------------------------------------------------------
\cvsection{Atividades extracurriculares}


%-------------------------------------------------------------------------------
%	CONTENT
%-------------------------------------------------------------------------------
\begin{cventries}

%---------------------------------------------------------
\cventry
{Projeto de férias} % Affiliation/role
{cTwittLike (web e aplicativo iOS)} % Organization/group
{Google AppEngine, Python, Swift, App Store} % Location
{2009 - 2014} % Date(s)
{
  \begin{cvitems} % Description(s) of experience/contributions/knowledge
    \item {Aplicativo com o objetivo de visualizar a timeline do Twitter na visão de outro usuário.}
    \item {Primeiramente desenvolvido em Python e executado no Google AppEngine.}
    \item {Aparições no \href{https://mashable.com/archive/ctwitterlike}{Mashable}, \href{https://www.engadget.com/2014-10-08-view-someone-elses-own-twitter-timeline-with-ctwittlike.html}{Engadget}, Tecnoblog, entre outros, causando grande tráfego e trabalhando a escala do projeto para atender as requisições.}
    \item {Anos mais tarde a ideia foi desenvolvida em Swift e disponibilizado como aplicativo para iOS.}      
  \end{cvitems}
}


%---------------------------------------------------------
\cventry
{Projeto de férias} % Affiliation/role
{Reminder Widget: for Evernote Reminders (aplicativo iOS)} % Organization/group
{Objective-C} % Location
{2015} % Date(s)
{
\begin{cvitems} % Description(s) of experience/contributions/knowledge
  \item {App criado para suprir a falta de um widget (melhor que o oficial na época) para o iOS que tivesse os reminders do Evernote.}
  \item {Desenvolvido logo na disponibilização do desenvolvimento de widgets para iOS.}
  \item {Detalhes no \href{https://appadvice.com/app/reminder-widget-for-evernote-reminders/1000192150}{AppAdvice}.}
\end{cvitems}
}


%---------------------------------------------------------
  \cventry
    {Projeto de férias} % Affiliation/role
    {Ganhei? (Aplicativo iOS)} % Organization/group
    {Swift, Node.js, AWS, App Store} % Location
    {2017} % Date(s)
    {
      \begin{cvitems} % Description(s) of experience/contributions/knowledge
        \item {App para cadastrar jogos e verificar resultados da Mega-Sena.}
        \item {Projeto desenvolvido em Swift com backend em Node.js com entrega de notificações ao sair o resultado do sorteio.}
        \item {Uso de OCR para reconhecimento dos números jogados ao ler o bilhete da aposta e cadastro automático.}
        \item {Saiu no site \href{https://macmagazine.com.br/post/2017/06/14/com-o-app-ganhei-voce-cadastra-seus-jogos-da-mega-sena-com-uma-foto-e-sabe-se-ficou-milionario-em-primeira-mao/}{MacMagazine} como sugestão de aplicativo.}
      \end{cvitems}
    }


%---------------------------------------------------------
\end{cventries}
